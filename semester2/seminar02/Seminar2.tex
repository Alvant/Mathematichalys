\documentclass[a4paper,12pt]{article}

\usepackage{mystyle}

\usepackage{gensymb}
\usepackage{scalerel}
\usepackage{stackengine}

% \usepackage{skull}  % skull
\usepackage{halloweenmath}  % \bigpumpkin, skull (https://tug.ctan.org/info/symbols/comprehensive/symbols-a4.pdf -- Table 76)

\usepackage{tikzsymbols}

% https://tex.stackexchange.com/questions/3266/how-do-i-use-a-circle-as-a-math-accent-larger-than-mathring
% https://tex.stackexchange.com/a/3270/135045
\usepackage{accents}

\renewcommand{\mathring}[1]{\accentset{\circ}{#1}}


\graphicspath{ {images/} }


% https://tex.stackexchange.com/questions/5461/is-it-possible-to-change-the-size-of-an-arrowhead-in-tikz-pgf
\usetikzlibrary{arrows.meta}


\DeclareMathOperator{\Image}{Im}

\definecolor{pink}{RGB}{218, 3, 174}
\definecolor{violet}{RGB}{148, 0, 211}
\definecolor{green}{RGB}{0, 153, 0}
\definecolor{orange}{RGB}{255, 153, 0}
\definecolor{blue}{RGB}{5, 73, 255}
\definecolor{cyan}{RGB}{31, 206, 203}
\definecolor{cyan2}{RGB}{0, 166, 147}
\definecolor{cyangreen}{RGB}{0, 155, 118}
\definecolor{cyangreen2}{RGB}{0, 109, 91}


% https://tex.stackexchange.com/a/101138/135045

\newcommand\widesim[1]{\ThisStyle{%
  \setbox0=\hbox{$\SavedStyle#1$}%
  \stackengine{-.1\LMpt}{$\SavedStyle#1$}{%
    \stretchto{\scaleto{\SavedStyle\mkern.2mu\sim}{.5150\wd0}}{.6\ht0}%
  }{O}{c}{F}{T}{S}%
}}


\newcommand{\BigMiddleThree}{\;\left|\vphantom{\begin{pmatrix} 0\\0\\0 \end{pmatrix}}\right.\;}
\newcommand{\BigMiddleFour}{\;\left|\vphantom{\begin{pmatrix} 0\\0\\0\\0 \end{pmatrix}}\right.\;}


% https://tex.stackexchange.com/questions/63531/how-to-write-quotation-marks-in-math-environment
\DeclareMathSymbol{\mlq}{\mathord}{operators}{``}
\DeclareMathSymbol{\mrq}{\mathord}{operators}{`'}

% TODO: didn't work...
% \renewcommand{\Re}{\mathop{\mathrm{Re}}\nolimits}
% \renewcommand{\Im}{\mathop{\mathrm{Im}}\nolimits}

\DeclareMathOperator{\Arg}{Arg}
\DeclareMathOperator{\Real}{Re}
\DeclareMathOperator{\Imag}{Im}


% https://tex.stackexchange.com/questions/544453/undefined-control-sequence-after-paragraph
\renewcommand{\paragraph}[1]{\noindent\textbf{#1}\quad}


% https://tex.stackexchange.com/questions/36851/skipping-line-after-proof-in-proof-environment#comment73553_36851
\newcommand{\proofindent}{\hspace*{\fill}\par\vspace{0.5em}\noindent}


% https://tex.stackexchange.com/questions/4813/extendible-equals-sign
\makeatletter
\newcommand*{\Relbarfill@}{\arrowfill@\Relbar\Relbar\Relbar}
\newcommand*{\xeq}[2][]{\ext@arrow 0055\Relbarfill@{#1}{#2}}
\makeatother


% https://tex.stackexchange.com/questions/142763/strikethrough-cyrillic-text-in-xetex-preserving-hyphenation
% https://tex.stackexchange.com/a/142764/135045
\makeatletter
\font\SOUL@tt="Palatino Linotype"
\setbox\z@\hbox{\SOUL@tt-}
\SOUL@ttwidth\wd\z@
\makeatother


% https://tex.stackexchange.com/questions/279100/typeset-the-shrug-%C2%AF-%E3%83%84-%C2%AF-emoji
\newcommand{\shrug}[1][]{%
\begin{tikzpicture}[baseline,x=0.8\ht\strutbox,y=0.8\ht\strutbox,line width=0.125ex,#1]
  \def\arm{(-2.5,0.95) to (-2,0.95) (-1.9,1) to (-1.5,0) (-1.35,0) to (-0.8,0)};
  \draw \arm;
  \draw[xscale=-1] \arm;
  \def\headpart{(0.6,0) arc[start angle=-40, end angle=40,x radius=0.6,y radius=0.8]};
  \draw \headpart;
  \draw[xscale=-1] \headpart;
  \def\eye{(-0.075,0.15) .. controls (0.02,0) .. (0.075,-0.15)};
  \draw[shift={(-0.3,0.8)}] \eye;
  \draw[shift={(0,0.85)}] \eye;
  % draw mouth
  \draw (-0.1,0.2) to [out=15,in=-100] (0.4,0.95); 
\end{tikzpicture}}



% https://tex.stackexchange.com/a/314638/135045
% \newcommand{\diff}{\mathop{}\!d\!}
\newcommand{\diff}{\mathop{}\!d}


% https://tex.stackexchange.com/questions/387570/how-to-make-cdot-operator-same-width-as-division-slash-operator-and-vice-ve
\newcommand*{\slashdiv}{\makebox[\widthof{${}\cdot{}$}]{{}/{}}}


% https://tex.stackexchange.com/questions/9641/filled-diamondsuit-and-heartsuit
% https://tex.stackexchange.com/a/9643/135045
\DeclareSymbolFont{extraup}{U}{zavm}{m}{n}
\DeclareMathSymbol{\varheartsuit}{\mathalpha}{extraup}{86}
\DeclareMathSymbol{\vardiamond}{\mathalpha}{extraup}{87}
% https://tex.stackexchange.com/questions/234942/whats-the-best-way-to-make-a-heart-butt-in-latex
\newcommand{\heart}{\ensuremath\varheartsuit}


\author{Алексеев Василий}


\title{Семинар 2}
\date{14 февраля ($\heart$) 2025}


\begin{document}
  \maketitle
  
  \tableofcontents

  \thispagestyle{empty}
  
  \newpage
  
  
  
  \vspace*{\fill}
  
  \noindent
  \emph{
    К формулировкам и доказательствам (если такие вообще приводятся) стоит относиться критически.
    Основное в этом конспекте~---~решение задач (но ``критичность'' и здесь лучше не отключать).
    За строгой, ясной и последовательной теорией лучше обращаться к ``нормальным'' источникам.
    (Например, к лекциям.)
  }
  
  \vspace*{\fill}
  
  \thispagestyle{empty}
  
  \newpage
  
  
  \pagenumbering{arabic}

  \section{Неопределённый интеграл (продолжение)}

  \subsection{Интегралы от рациональных функций (возвращение)}

  Рассмотрим пример интеграла от \emph{неправильной} ``многочленной'' дроби.
  
  \subsubsection{С2, \S 2, \textnumero 2(7)}

  Найти интеграл:
  \begin{equation}\label{eq:2-2(7)-int}
    J = \int \frac{x^5 + x^4 - 8}{x^3 - 4x} \diff x
  \end{equation}
  
  \begin{solution}
    Перед тем как начать искать этот интеграл по ``алгоритму'' (разложить на множители знаменатель, расписать дробь как сумму дробей, найти коэффициенты, перейти от интеграла суммы к сумме интегралов), надо сначала выделить ``целую часть'', оставив в числителе ``остаток'' от деления (многочлен меньшей степени, чем знаменатель):
    \[
      x^5 + x^4 - 8 = q(x) \cdot (x^3 - 4x) + r(x),\quad \deg r(x) < \deg(x^3 - 4x) = 3
    \]
    \[
      \Rightarrow J = \int q(x) \diff x + \int \frac{r(x)}{x^3 - 4x} \diff x = \ldots
    \]

    Как выделить целую часть из ``многочленной'' дроби?
    Идея такая же, как с обычными числовыми дробями~---~поделим числитель на знаменатель.
    Будем делить в столбик, на каждой шаге подбирая слагаемое в частном так, чтобы в результате ``уничтожить'' текущий член с $x$ в максимальной степени в делимом.
    \begin{table}[ht]
        \centering
        
        \caption{Деление многочлена $x^5 \hm+ x^4 \hm- 8$ на $x^3 \hm- 4x$ ``в столбик'' (по техническим причинам нарисовано в виде таблицы).}
        \label{tab:my_label}
        
        \begin{tabular}{l|l|l|l}
             \toprule
             Делимое           & Делитель   & Частное & Остаток\\
             \midrule
             $x^5 + x^4 - 8$   & $x^3 - 4x$ & $x^2$   & $x^4 + 4x^3 - 8$\\
             $x^4 + 4x^3 - 8$  & $x^3 - 4x$ & $x$     & $4x^3 + 4x^2 - 8$\\
             $4x^3 + 4x^2 - 8$ & $x^3 - 4x$ & $4$     & $4x^2 + 16x - 8$\\
             \bottomrule
        \end{tabular}
    \end{table}

    Итого:
    \[
      x^5 + x^4 - 8 = (x^2 + x + 4) + \frac{4x^2 + 16x - 8}{x^3 - 4x}
    \]

    И далее задача уже сводится к поиску интеграла от правильной дроби:
    \[
      J = \int (x^2 + x + 4) \diff x + 4 \int \frac{x^2 + 4x - 2}{x(x - 2)(x + 2)}
    \]

    Не будем доводить номер до конца~---~оставим эту возможность интересующемуся читателю при желании в качестве упражнения.
    (А неинтересующемуся в любом случае не важно $\mathwitch*$).
  \end{solution}


  \subsection{Интегралы от иррациональных функций (продолжение)}

  Познакомимся с ещё одним приёмом взятия интегралов от иррациональных функций.
  Это~---~замены на $\cos t$, $\sin t$, $\ch t$, $\sh t$ или ещё на некоторые тригонометрические/гиперболические функции\footnote{  % TODO: hyphen after / in тригонометрические/гиперболические
    Подробнее см. теорсправку в С2, пар. 3.
  } с тем, чтобы благодаря тригонометрическим/гиперболическим тождествам с квадратами иррациональность исчезла.

  \subsubsection{Пример}
  
  \begin{example}
    Найдём интеграл:
    \begin{equation}\label{eq:sqrt-example-int}
      J = \int \sqrt{x^2 - 1} \diff x
    \end{equation}

    На выражение под корнем можно смотреть как на разность квадратов $x$ и единицы.
    В каком случае разность квадратов может давать квадрат?

    \medskip

    \noindent
    \emph{Способ 1 (замена на тригонометрическое)}.
    Область определения функции под интегралом: $|x| \hm\geq 1$.
    
    Положим $x \hm\equiv \frac{1}{\cos t}$.
    Для взаимной однозначности замены будем считать $t \hm\in [0, \pi]$.
    Учитывая, что должно также выполняться $\cos t \hm{\not=} 0$, получаем $t \in [0, \pi/2) \hm\cup (\pi/2, 1]$.
    Видно, что $t$, приводящие к положительным~$x$, ``отделены'' от~$t$, проводящих к отрицательным.
    (Область допустимых значений~$t$ как бы ``разрезана'' посередине.)
    Чтобы подстраховаться от каких-либо возможных осложнений в этой связи в будущем, рассмотрим для простоты $x \hm> 0$ и соответствующие им $t \hm\in [0, \pi/2)$.
    % TODO: сослаться на какой-нибудь нормальный источник
    (В идеале после этого надо бы было рассмотреть отдельно случай и отрицательных~$x$, но можно убедиться, что получится то же самое~---~появляющиеся в разных местах ``минусы'' друг друга в итоге скомпенсируют.)
    
    Итого, при такой замене получим:
    \[
      x^2 - 1 = \frac{1}{\cos^2 t} - 1 = \frac{\sin^2 t}{\cos^2 t} = \tg^2 t
    \]
    Таким образом, иррациональность уйдёт!
    Для проведения замены в интеграле найдём также выражение $\diff x$ через~$t$:
    \[
      \diff x = \diff\left(\frac{1}{\cos t}\right) = -\frac{1}{\cos^2 t} \cdot (-\sin t) \diff t
    \]

    В итоге в результате замены получаем такой интеграл:\footnote{
      Так как $t \hm\in [0, \pi/2)$, то $\sqrt{\tg^2 t} = \tg t$ без модуля.
    }
    \[
      J = \int \tg t \cdot \frac{\sin t}{\cos^2 t} \diff t = \blacktriangle
    \]

    Корня больше нет, зато теперь под интегралом тригонометрические функции...
    Можно попробовать расписать тангенс:
    \[
      \blacktriangle = \int \frac{\sin t}{\cos t} \frac{\sin t}{\cos^2 t} \diff t = \int \frac{\sin^2 t}{\cos^3 t} \diff t = \diamondsuit
    \]

    Теперь $\sin^2 t$ можно выразить через $\cos^2 t$~---~в результате под интегралом получится функция от $\cos t$.
    Если бы можно было сделать замену, то получился бы интеграл от рациональной функции...
    (Пока же замену сделать нельзя, потому что ещё остаётся $\diff x$.)
    Однако можно попробовать ещё один ``приём'', который иногда помогает с интегралами от тригонометрических~---~домножим и поделим, например, на $\cos t$:
    \[
      \diamondsuit = \int \frac{\sin^2 t}{\cos^3 t} \frac{\cos t \diff t}{\cos t}
        = \int \frac{\sin^2 t}{\cos^4 t} \diff \sin t = \spadesuit
    \]

    Что получилось?
    В числителе $\sin^2 t$, в знаменателе $\cos^4 t$ (выражающийся через $\sin^2 t$), и всё это по $\diff \sin t$!
    Таким образом, получается замена $u \hm\equiv \sin t$ (очевидно, $|u| \hm\leq 1$, но также и $u \hm> 0$~---~при рассматриваемом $t$) приводит к интегралу от рациональной дроби.
    Находим его и возвращаемся обратно к~$x$:\footnote{
      В одном из последних переходов используется введённое в самом начале ограничение на знак $x$: $\sqrt{x^2} \hm= x$ без модуля.
    }
    \begin{equation*}
    \begin{split}
      \spadesuit &= \int \frac{u^2}{(1 - u^2)^2} \diff u\\
        &= \ldots\\
        &= \frac{u}{2(1 - u^2)} + \frac{1}{4} \ln\left|1 - u\right| - \frac{1}{4} \ln\left|1 + u\right| + C\\
        &=  \frac{\sin t}{2(1 - \sin^2 t)} + \frac{1}{4} \ln\left(\frac{1 - \sin t}{1 + \sin t}\right) + C\\
        &= \frac{\sqrt{1 - \cos^2 t}}{2 \cos^2 t} + \frac{1}{4} \ln\left(\frac{1 - \sqrt{1 - \cos^2 t}}{1 + \sqrt{1 - \cos^2 t}}\right) + C\\
        &= \frac{x \sqrt{x^2 - 1}}{2} + \frac{1}{4} \ln\left|\frac{x - \sqrt{x^2 - 1}}{x + \sqrt{x^2 - 1}}\right| + C
    \end{split}
    \end{equation*}

    Но рассмотренная замена~---~не единственный возможный способ решения (и не единственная возможная ``решающая интеграл'' замена).

    \medskip

    \noindent
    \emph{Способ 2 (замена на гиперболическое)}.

    Вспомним гиперболические синус и косинус:
    \begin{equation}\label{eq:sh-and-ch}
      \begin{aligned}
        &\sh x = \frac{e^x - e^{-x}}{2}\\
        &\ch x = \frac{e^x + e^{-x}}{2}
      \end{aligned}
    \end{equation}

    И ``основное'' тождество с ними:\footnote{
      Проверяется просто подстановкой из формул~\eqref{eq:sh-and-ch}.
    }
    \[
      \ch^2 x - \sh^2 x = 1
    \]

    Вспоминая про интеграл~\eqref{eq:sqrt-example-int}, можем увидеть такой вариант избавляющей от корня замены:\footnote{
      При такой замене неявно налагается также ограничение и на $x$: раз $x \hm= \ch t$, то $x \hm\geq 1$.
      В идеале стоило бы потом рассмотреть ещё замену $x \hm= -\ch t$, чтобы учесть $x \hm\leq -1$.
      Однако, как и в прошлый раз, несложно будет убедиться в том, что в конечном результате ничего не поменяется.
    }
    \[
      \begin{aligned}
        &x \equiv \ch t,\quad t > 0\ \mbox{(для определённости)}\\
        &\diff x = \sh t \diff t
      \end{aligned}
    \]

    В результате приходим к интегралу:\footnote{
      При $t \hm> 0$ будет $\sqrt{\sh^2 t} \hm= \sh t$ без модуля.
    }
    \[
      J = \int \sh t \cdot \sh t \diff t = \int \sh^2 t \diff t
    \]

    Этот интеграл можно (как и аналогичные тригонометрические) взять с помощью понижения степени:
    \[
      \left.
        \begin{aligned}
          &\sh^2 t = \frac{e^{2t} + e^{-2t} - 2}{4}\\
          &\ch^2 t = \frac{e^{2t} + e^{-2t} + 2}{4}\\
          &\ch 2t = \frac{e^{2t} + e^{-2t}}{2}
        \end{aligned}
      \right|
      \Rightarrow \begin{aligned}
        &\vphantom{\frac{\ch 2t - 1}{2}}\\
        &\ch 2t = \ch^2 t + \sh^2 t\\
        &\sh^2t = \frac{\ch 2t - 1}{2}
      \end{aligned}
      % \Rightarrow \sh^2t = \frac{\ch 2t - 1}{2}
    \]

    Таким образом,
    \begin{equation*}
    \begin{split}
      J &= \int \frac{\ch 2t - 1}{2} \diff t\\
        &= \frac{1}{2} \left(\int \ch 2t \diff t - \int \diff t\right)\\
        &= \frac{1}{2} \left(\frac{1}{2} \sh 2t - t\right) + C = \blacktriangle
    \end{split}
    \end{equation*}

    Чтобы вернуться обратно к $x$, выразим
    \[
      \sh 2t = 2 \sh t \ch t = 2 \sqrt{\ch^2 t - 1} \ch t = 2 \sqrt{x^2 - 1} \cdot x
    \]

    Также надо выразить $t$ через $x$...\footnote{
      Далее делаем по аналогии с нахождением ``длинного логарифма'' из прошлого конспекта.
    }
    \[
      \begin{aligned}
        &x = \ch t,\quad t > 0\\
        &x = \frac{1}{2}\left(e^t + e^{-t}\right)  % = \frac{1}{2}\left(z + \frac{1}{z}\right),\quad z \equiv e^t > 1
      \end{aligned}
    \]
    %\[
      %\begin{aligned}
        %&z^2 -2x \cdot z + 1 = 0\\
        %&z_{1, 2} = x \pm \sqrt{x^2 - 1}  % \frac{x \pm \sqrt{x^2 - 1}}{1}
      %\end{aligned}
    %\]
    
    Если ввести ещё одну замену $z \hm\equiv e^t \hm> 1$, то можно прийти к квадратному уравнению относительно~$z$:
    \[
      \begin{aligned}
        &x = \frac{1}{2}\left(z + \frac{1}{z}\right)\\
        &z^2 -2x \cdot z + 1 = 0 \Rightarrow z_{1, 2} = x \pm \sqrt{x^2 - 1}
      \end{aligned}
    \]

    Раз положили $t \hm> 0$, то, очевидно, нужен больший из двух~$z$, поэтому в итоге:
    \[
      \begin{aligned}
        &z = x + \sqrt{x^2 - 1}\\
        &t = \ln z = \ln \left(x + \sqrt{x^2 - 1}\right)
      \end{aligned}
    \]

    Наконец досчитываем интеграл (вспоминая также о том, что $x$ может быть и отрицательный):
    \[
      \blacktriangle = \frac{x \sqrt{x^2 - 1}}{2} - \frac{1}{2} \ln \left|x + \sqrt{x^2 - 1}\right| + C
    \]

    (На первый взгляд, ответ отличается от полученного прошлым способом, но это не так.)

    \medskip

    \noindent
    \emph{Способ 3 (замена дроби): Эскиз}.

    На примере этого интеграла можно вспомнить и ещё один вариант замены в интегралах с корнями:\footnote{
      Чтобы что-то сделать с образовавшимся модулем, можно бы было при желании попробовать следующее: просто убрать модуль под интегралом, считая как бы $x \hm+ 1 \hm> 0$.
      И потом в полученной формуле для первообразной заменить все скобки $(x \hm+ 1)$ на $\sqrt{(x + 1)^2}$.
      Также стоило бы проследить, чтобы в первообразной $(x \hm+ 1)$ не осталась ни в каком знаменателе (ведь в исходной подынтегральной функции точка $x \hm= -1$ допускается.)
    }
    \[
      J = \int \sqrt{(x - 1)(x + 1)} \diff x = \int |x + 1| \sqrt{\frac{x - 1}{x + 1}} \diff x
    \]

    \medskip

    \noindent
    \emph{Способ 4 (по старинке~---~по частям)}.

    Можно попробовать интегрировать без каких-то замен, а просто по частям:
    \[
      \int u \diff v = uv - \int v \diff u
    \]

    В данном случае: $u \hm= \sqrt{x^2 - 1}$ и $v \hm= x$:
    \begin{equation}
    \begin{split}
      J &= \int \sqrt{x^2 - 1} \diff x\\
        &= x \sqrt{x^2 - 1} - \int x \cdot \diff \sqrt{x^2 - 1}\\
        &= x\sqrt{x^2 - 1} - \int \frac{x^2}{\sqrt{x^2 - 1}} \diff x\\
        &= x\sqrt{x^2 - 1} - \int \frac{x^2 - 1}{\sqrt{x^2 - 1}} \diff x - \int \frac{1}{\sqrt{x^2 - 1}} \diff x\\
        &= x\sqrt{x^2 - 1} - J - \ln|x + \sqrt{x^2 - 1}|
    \end{split}
    \end{equation}
    Видно, что в последнем переходе получается точно такой же интеграл, как в начале.
    В результате получается как бы уравнение относительно~$J$.
    (Правда, это всё же не совсем уравнение~---~ведь $J$ это \emph{совокупность} первообразных):
    \[
      J = x\sqrt{x^2 - 1} - J - \ln\left|x + \sqrt{x^2 - 1}\right|
    \]

    Пожалуй, точнее было бы записать так:
    \[
      F(x) + P = x\sqrt{x^2 - 1} - (F(x) + Q) - \ln\left|x + \sqrt{x^2 - 1}\right|
    \]
    где $F(x)$~---~первообразная, а $P, Q \hm\in \RR$.

    В итоге получаем:
    \[
      F(x) = \frac{x\sqrt{x^2 - 1}}{2} - \frac{1}{2}\ln\left|x + \sqrt{x^2 - 1}\right| + C
    \]

    Так выглядит произвольная первообразная подынтегральной функции.
    Очевидно, так же можно записать и неопределённый интеграл.
  \end{example}

  % \begin{example}
  %   Найдём интеграл:
  %   \[
  %     \int x \sqrt{x^2 - 1} \diff x
  %   \]
  % \end{example}

  \subsubsection{С2, \S 3, \textnumero 18(3)}

  Найти интеграл:
  \[
    J = \int \frac{
      \sqrt[3]{1 + \sqrt[4]{x}}
    }{
      \sqrt{x}
    } \diff x
  \]
  
  \begin{solution}
    В функции под интегралом есть несколько корней.
    Причём $\sqrt x \hm= \left(\sqrt[4]{x}\right)^2$.
    Какую замену можно провести, чтобы уйти от корней? (хотя бы от некоторых)
    \[
      \begin{aligned}
        &\sqrt[4]{x} \equiv t \geq 0\\
        &x = t^4\\
        &\diff x = 4t^3 \diff t
      \end{aligned}
    \]

    Интеграл при этом:
    \[
      J = \int \frac{\sqrt[3]{1 + t}}{t^2} \cdot 4t^3 \diff t = 4 \int t \sqrt[3]{1 + t} \diff t
    \]

    Что теперь?
    Можно сделать ещё одну замену:
    \[
      \begin{aligned}
        &1 + t \equiv u^3,\quad u \in \RR\\
        &\diff t = 3u^2 \diff u
      \end{aligned}
    \]

    Интеграл в результате замены:
    \[
      J = 4 \int (u^3 - 1) u \cdot 3u^2 \diff u = \frac{12}{7} u^7 - 3u^4 + C = \blacktriangle
    \]

    И обратные замены:
    \begin{equation*}
    \begin{split}
      \blacktriangle &= \frac{12}{7} (1 + t)^{7/3} - 3(1 + t)^{4/3} + C\\
        &= \frac{12}{7} (1 + t)^{7/3} - 3(1 + t)^{4/3} + C\\
        &= \frac{12}{7} \left(1 + \sqrt[4]{x}\right)^{7/3} - 3 \left(1 + \sqrt[4]{x}\right)^{4/3} + C
    \end{split}
    \end{equation*}
  \end{solution}


  \subsection{Интегралы от трансцендентных (не алгебраических) функций}

  Интегралы от чего-то кроме рациональных дробей и корней (тригонометрические, гиперболические функции, показательная, логарифм).

  \subsubsection{С2, \S 4, \textnumero 3(1)}

  Найти интеграл:
  \begin{equation}
    J = \int \frac{\sin 2x}{\cos^3 x} \diff x
  \end{equation}
  
  \begin{solution}
    Решение из серии ``не знаешь, что делать~---~делай, что можешь'': распишем синус двойного угла и посмотрим, что получится:
    \[
      J = \int \frac{2 \sin x \cos x}{\cos^3 x} \diff x = 2 \int \frac{\sin x}{\cos^2 x} \diff x
    \]

    Теперь можно, воспользовавшись основным тригонометрических тождеством, выразить $\cos^2 x$ через синус...
    Можно, но это ни к чему не приведёт: в функции под интегралом будет только синус, но замены ($\sin x \hm\equiv t$) сделать не получится, потому что под дифференциалом-то стоит не синус.

    Но можно занести синус под дифференциал:
    \[
      \sin x \diff x = \diff (-\cos x) = -\diff \cos x
    \]

    Это приведёт к тому, что в интеграле проявится естественная замена $\cos x \hm\equiv t$, которая сделает из него интеграл от рациональной функции:
    \[
      J = -2\int \frac{\diff \cos x}{\cos^2 x} \xrightarrow[]{\cos x \equiv t} -2\int \frac{\diff t}{t^2} = \frac{2}{t} + C = \frac{2}{\cos x} + C
    \]
  \end{solution}


  \subsubsection{С2, \S 4, \textnumero 4(2)}

  Найти интеграл:
  \begin{equation}
    J = \int \sin^3 x \cos^4 x \diff x
  \end{equation}
  
  \begin{solution}
    Этот интеграл можно взять таким же приёмом: ``откусыванием'' $\sin x$, занесением его под дифференциал (он при этом перейдёт в косинус) и заменой (под интегралом к этому моменту останется косинус и квадрат синуса, также выражающийся через косинус):
    \[
      J = \int \underbrace{\:\sin^2 x}_{1 - \cos^2 x} \cdot \cos^4 x \cdot \overbrace{\sin x \diff x}^{-\diff \cos x} \xrightarrow{\cos x \equiv t} = -\frac{t^5}{5} + \frac{t^7}{7} + C = \frac{\cos^7 x}{7} - \frac{\cos^5 x}{5} + C
    \]

    % TODO: fail...
    % \medskip

    % \noindent
    % \emph{Другой способ (другая замена)}.

    % Можно заметить, что если ``откусить'' один косинус (засунуть его под дифференциал), то то, что останется, можно будет свернуть в синус двойного угла.
    % Далее можно будет попробовать взять интеграл по частям (никакой другой возможности в результате и не останется):
    % \[
    %   J = \int \underbrace{\:\sin^3 x \cos^3x}_{\frac{1}{2^3} \sin^3 2x} \cdot \overbrace{\cos x \diff x}^{\diff \sin x} = \frac{1}{8} \left(\sin^3 2x \cdot \sin x - \int \sin x \diff \sin^3 2x\right) = \spadesuit
    % \]

    % Посмотрим отдельно получившийся в результате ``по частям'' интеграл:
    % \[
    %   \int \sin x \diff \sin^3 2x = \int \sin x \cdot \overbrace{\left(3\sin^2 2x \cdot \cos 2x \cdot 2\right)}^{(\sin^3 2x)'} \diff x
    % \]
  \end{solution}


  \subsubsection{``Лемма-номер''}\label{seq:lemma-cos}

  Найти интеграл:
  \begin{equation}
    \int \frac{1}{\cos x} \diff x
  \end{equation}
  
  \begin{solution}
    Воспользуемся \emph{универсальной тригонометрической подстановкой}~---~это выражение $\sin x$ и $\cos x$ через тангенс половинного угла $\tg \frac{x}{2}$~---~с тем чтобы потом ввести замену на тангенс.
    В результате этой замены интеграл с тригонометрическими функциями можно свести к интегралу от рациональной функции.
    Правда, вычислений такое интегрирование может занимать не мало.
    (Поэтому если есть ещё идеи, что делать с интегралом, то универсальную тригонометрическую лучше приберегать как запасной вариант, особенно если синус и косинус присутствуют в степенях больше единицы.)

    Выведем формулы для проведения замены.
    Для этого смотрим на $\sin x$ и $\cos x$ как на функции ``двойного половинного аргумента'', а потом ещё на единицу~---~как на сумму квадратов синуса и косинуса:
    \begin{equation*}
    \begin{split}
      \sin x &= \sin{\left(2 \cdot \frac{x}{2}\right)}\\
        &= 2 \sin \frac{x}{2} \cos \frac{x}{2}\\
        &= \frac{2 \sin \frac{x}{2} \cos \frac{x}{2}}{\sin^2 \frac{x}{2} + \cos^2 \frac{x}{2}}
        = \frac{2 \tg \frac{x}{2}}{\tg^2 \frac{x}{2} + 1}
    \end{split}
    \end{equation*}
    \begin{equation*}
    \begin{split}
      \cos x &= \cos{\left(2 \cdot \frac{x}{2}\right)}\\
        &= \cos^2 \frac{x}{2} - \sin^2 \frac{x}{2}\\
        &= \frac{\cos^2 \frac{x}{2} - \sin^2 \frac{x}{2}}{\sin^2 \frac{x}{2} + \cos^2 \frac{x}{2}}
        = \frac{1 - \tg^2 \frac{x}{2}}{\tg^2 \frac{x}{2} + 1}
    \end{split}
    \end{equation*}

    Заменяя $\tg \frac{x}{2} \hm\equiv t$, $t \hm\in \RR$, получаем:
    \begin{equation}
      \begin{aligned}
        &\sin x = \frac{2 t}{1 + t^2}\\
        &\cos x = \frac{1 - t^2}{1 + t^2}
      \end{aligned}
    \end{equation}

    Раз $t \hm= \tg \frac{x}{2}$, то $x \hm= 2 \arctg t$, поэтому для дифференциала $\diff x$ получаем такое выражение через~$t$:
    \[
      \diff x = \diff (2 \arctg t) = \frac{2}{1 + t^2} \diff t
    \]

    В итоге можем посмотреть на интеграл после замены:
    \begin{equation*}
    \begin{split}
      J &= \int \frac{1}{\frac{1 - t^2}{1 + t^2}} \cdot \frac{2}{1 + t^2} \diff t\\
        &= 2 \int \frac{1}{1 - t^2} \diff t\\
        &= \int \left(\frac{1}{1 - t} + \frac{1}{1 + t}\right) \diff t\\
        &= -\ln|1 - t| + \ln |1 + t| + C
        = \ln \left|\frac{1 + t}{1 - t}\right| + C
    \end{split}
    \end{equation*}

    Возвращаясь обратно к~$x$, получаем ответ:
    \[
      J = \ln \left|\frac{1 + \tg \frac{x}{2}}{1 - \tg \frac{x}{2}}\right| + C
    \]

    Рассмотрим интереса ради ещё один способ нахождения этого интеграла.

    \medskip

    \noindent
    \emph{Ещё один \st{трюк} способ}.

    Вспомним интеграл:
    \[
      J = \int \frac{1}{\cos x} \diff x
    \]

    Домножим и поделим дробь под интегралом на $\cos x$.
    При этом ничего по сути не изменится.
    Однако в результате появится возможность смотреть на $\cos x \diff x$ сверху как на дифференциал синуса $\diff \sin x$, а в знаменателе будет $\cos^2 x$~---~тоже выражающийся через синус как $1 \hm- \sin^2 x$.
    А это ``напрашивающаяся'' замена синуса:
    \begin{equation*}
    \begin{split}
      J &= \int \frac{1}{\cos x} \cdot \frac{\cos x}{\cos x} \cdot \diff x\\
        &= \int \frac{1}{\cos^2 x} \cdot (\cos x\diff x)\\
        &= \int \frac{1}{1 - \sin^2 x} \diff \sin x
        % &= \int \frac{1}{1 - t^2} \diff t \quad\quad (\sin x \equiv t)\\
        = \frac{1}{2} \ln\left|\frac{1 + \sin x}{1 - \sin x}\right| + C
    \end{split}
    \end{equation*}

    (Ответ получился такой же, как в результате универсальной тригонометрической подстановки.)
  \end{solution}
  

  \subsubsection{С2, \S 4, \textnumero 9(1)}

  Найти интеграл:
  \begin{equation}
    \int \frac{1}{\cos^3 x} \diff x
  \end{equation}
  
  \begin{solution}
    Попробуем подойти к этому интегралу так же, как к предыдущему попроще~(\ref{seq:lemma-cos}).

    \medskip

    \noindent
    \emph{Способ 1: ``Универсальный''}.

    Применять универсальную тригонометрическую подстановку для этого номера, на первый взгляд, кажется, ``опасным'', потому $\tg \frac{x}{2}$ будет в квадрате, сам косинус при этом ещё в кубе, итого можно ожидать, что получится что-то в шестой степени.
    Но, возможно, что-нибудь сократится при умножении на преобразованный $\diff x$, и получится не так безысходно:
    \[
      J = \int \frac{1}{\left(\frac{1 - t^2}{1 + t^2}\right)^{\!3}} \cdot \frac{2}{1 + t^2} \diff t = 2\int \frac{(1 + t^2)^2}{(1 - t^2)^3} \diff t
    \]

    Ничего удивительного не произошло~---~шестая степень не исчезла.
    Если дальше пытаться искать разложение дроби под интегралом методом неопределённых коэффициентов, то этих коэффициентов будет шесть штук.
    Система шесть на шесть...
    Оставим желающим возможность довести это решение до конца, и перейдём к другому способу нахождения интеграла.

    \medskip

    \noindent
    \emph{Способ 2: ``Домножить-поделить-ничего-не-изменить''}.

    Домножим и поделим на косинус~---~снова косинус сверху можно будет занести под дифференциал, придя к синусу; а снизу степень косинуса дополнится до четвёртой~---~это можно будет представить как квадрат квадрата, а сам квадрат тоже можно будет выразить через синус.
    Получается в явном виде замена синуса:
    \begin{equation*}
    \begin{split}
      J &= \int \frac{1}{\textcolor{blue}{\cos^3 x}} \cdot \frac{\textcolor{pink}{\cos x}}{\textcolor{blue}{\cos x}} \cdot \textcolor{pink}{\diff x}\\
        &= \int \frac{1}{(1 - \sin^2 x)^2} \diff \sin x \xrightarrow{\sin x \equiv t} \int \frac{1}{(1 - t^2)^2} \diff t
    \end{split}
    \end{equation*}

    Кажется, что получившийся интеграл от рациональной приятнее, чем в способе с универсальной подстановкой: степень знаменателя четвёртая вместо шестой (надо будет искать меньше коэффициентов), сверху тоже всего лишь единица вместо многочлена с~$t$ (искать будет проще).
    Этот интеграл уже вполне можно за разумное время довести до конца.
    Однако снова делать этого не будем, а переключимся на ещё один возможный\footnote{
      Замеченный автором конспекта.
    } способ взятия интеграла.

    \medskip

    \noindent
    \emph{Способ 3: ``Многоликая единица''}.

    Снизу в интеграле стоит косинус в кубе.
    Естественным было бы желание каким-то образом это упростить~---~понизить степень косинуса.
    Сделать это через табличную формулу понижения степени для куба, кажется, не вариант, потому что косинус стоит в знаменателе (проще от такого понижения не станет, а вот если бы $\cos^3 x$ был в числителе...)

    Но кое-какой выход есть.

    На единицу, стоящую в знаменателе, можно смотреть таким образом:
    \[
      1 = \sin^2 x + \cos^2 x
    \]
    если так её расписать, то получим сумму квадратов сверху, куб снизу~---~возможно, что-то упростится:
    \[
      J = \int \frac{\sin^2 x + \cos^2 x}{\cos^3 x} \diff x = \int \frac{\sin^2 x}{\cos^3 x} \diff x + \int \frac{1}{\cos x} \diff x
    \]

    Второй интеграл уже находили: дли него вполне можно применить универсальную тригонометрическую замену~(\ref{seq:lemma-cos}).
    С первым же интегралом можно воспользоваться приёмом ``домножить-поделить''~---~на косинус:
    \[
      \int \frac{\sin^2 x}{\cos^3 x} \diff x
        = \int \frac{\sin^2 x}{\textcolor{blue}{\cos^3 x}} \cdot \frac{\textcolor{pink}{\cos x}}{\textcolor{blue}{\cos x}} \cdot \textcolor{pink}{\diff x}
        = \int \frac{\sin^2 x}{(1 - \sin^2 x)^2} \diff \sin x
    \]
    % получится $\diff \sin x$, останется $\sin^3 x$ сверху, а снизу~---~будет $\cos^4 x$, который есть $(1 \hm- \sin^2 x)^2$.
    в результате вырисовывается замена синуса.

    Получилось примерно так же, как в предыдущем варианте решения, даже посложнее.
    Но про возможность ``переодевать'' единицу при интегрировании синусов и косинусов стоит иметь в виду (очевидно, не самый универсальный приём, но иногда может быть полезно).

    \medskip

    Заметим в конце, что, будь степень косинуса в знаменателе не первой или третьей, а второй, то интегрировать можно было бы намного проще:
    \[
      \int \frac{1}{\cos^2 x} \diff x = \int \diff \tg x = \tg x + C
    \]
  \end{solution}



  \subsubsection{С2, \S 4, \textnumero 15(2)}

  Найти интеграл:
  \begin{equation}\label{eq:int-15(2)}
    J = \int \frac{\cos x - \sin x}{\cos x + \sin x} \diff x
  \end{equation}
  
  \begin{solution}
    \emph{Способ 1: ``Универсальный''}.
    Хотя начинать с выражения всего через тангенс половинного угла~---~это \st{практически всегда} часто не самый оптимальный способ решения, но в данном случае начнём именно с него, потому что кажется, что ситуация подходящая (синусы и косинусы в первой степени; и в числителе, и в знаменателе~---~что-то может сократиться):
    \[
      J = \int \frac{\frac{1 - t^2}{1 + t^2} - \frac{2 t}{1 + t^2}}{\frac{1 - t^2}{1 + t^2} + \frac{2 t}{1 + t^2}} \cdot \frac{2}{1 + t^2} \diff t = 2\int \frac{-t^2 - 2t + 1}{(-t^2 + 2t + 1)(1 + t^2)} \diff t
    \]

    Ничего особо не сократилось.
    В знаменателе четвёртая степень.
    Решаемо, но... посмотрим другой способ.

    \medskip

    \noindent
    \emph{Способ 2: ``Более подходящий тангенс''}.

    Под интегралом есть и синус, и косинус.
    Но несложно заметить, что если поделить числитель и знаменатель дроби на, скажем, косинус~---~то в функции под интергалом останется по сути один тангенс:
    \[
      J = \int \frac{\cos x - \sin x}{\cos x + \sin x} \diff x
      = \int \frac{1 - \frac{\sin x}{\cos x}}{1 + \frac{\sin x}{\cos x}} \diff x
      = \int \frac{1 - \tg x}{1 + \tg x} \diff x = \spadesuit
    \]

    Но заменить тангенс пока как будто нельзя, ведь остаётся ещё $\diff x$.
    Попробуем его тоже ``причесать'':
    \begin{equation*}
    \begin{split}
      \diff x &= \frac{\cos^2 x}{\cos^2 x} \diff x\\
        &= \cos^2 x \cdot \left(\frac{1}{\cos^2 x} \diff x\right)\\
        &= \frac{\cos^2 x}{1} \cdot \left(\tg' x \diff x\right)\\
        &= \frac{\cos^2 x}{\sin^2 x + \cos^2 x} \cdot \diff \tg x
        = \frac{1}{1 + \tg^2 x} \diff \tg x
    \end{split}
    \end{equation*}
    где использовалось занесение под дифференциал производной тангенса\footnote{
      Которую, конечно, надо было ещё заметить...
    } и ``многоликость'' единицы.

    Можем вернуться к интегралу:
    \[
      \spadesuit = \int \frac{1 - \tg x}{1 + \tg x} \cdot \frac{1}{1 + \tg^2 x} \diff \tg x
    \]
    который уже берётся заменой тангенса: $\tg x \hm\equiv t$, $t \hm\in \RR$.

    \medskip

    \noindent
    \emph{Способ 3: ``Домножить-поделить 2''}.

    Дробь под интегралом~\eqref{eq:int-15(2)} есть дробь вида $\frac{a + b}{a - b}$~---~отношение суммы (разности) двух чего-то к их разности (сумме).
    Наверняка читатель уже сталкивался с таким ``трюком'', когда верх и низ дроби домножается на сумму или разность с тем, чтобы снизу получилась разность квадратов.
    В общем, в интеграле этого номера можно попробовать сделать то же самое:
    \[
      J = \int \frac{(\cos x - \sin x)\textcolor{pink}{(\cos x - \sin x)}}{(\cos x + \sin x)\textcolor{pink}{(\cos x - \sin x)}} \diff x
      = \int \frac{(\cos x - \sin x)^2}{\cos^2 x - \sin^2 x} \diff x = \blacktriangle
    \]

    Распишем квадрат разности сверху и попробуем ещё как-то попреобразовывать~---~имея в виду, что в знаменателе уже по сути стоит косинус двойного угла:
    \begin{equation*}
    \begin{split}
      \blacktriangle &= \int \frac{\cos^2 x + \sin^2 x - 2\sin x \cos x}{\cos 2x} \diff x\\
      &= \int \frac{1 - \sin 2x}{\cos 2x} \diff x
      = \int \frac{1}{\cos 2x} \diff x - \int \frac{\textcolor{pink}{\sin 2x}}{\cos 2x} \textcolor{pink}{\diff x}
      = \frac{1}{2} \int \frac{\diff (2x)}{\cos 2x} + \frac{1}{2} \int \frac{\diff \cos 2x}{\cos 2x}
    \end{split}
    \end{equation*}

    С первым интегралом уже сталкивались ранее в конспекте~(\ref{seq:lemma-cos}), а второй по сути табличный (если мысленно заменить $\cos 2x$).

    \medskip

    \noindent
    \emph{Способ 4: ``Заметим''}.\footnote{
      Особенно легко такое решение угадывается по виду ответа.
    }

    Заметим, что числитель в дроби под интегралом~\eqref{eq:int-15(2)}~---~это производная знаменателя:
    \[
      (\cos x + \sin x)' = -\sin x + \cos x
    \]

    Таким образом, можно сделать моментально решающую интеграл замену:
    \[
      J = \int \frac{\textcolor{pink}{(\cos x + \sin x)'}}{\cos x + \sin x} \textcolor{pink}{\diff x}
      = \int \frac{\diff (\cos x + \sin x)}{(\cos x + \sin x)}
      = \ln |\cos x + \sin x| + C
    \]
  \end{solution}


  \subsubsection{С2, \S 4, \textnumero 21(1)}

  Найти интеграл:
  \begin{equation}
    \int \frac{\diff x}{1 + 4 \cos x}
  \end{equation}
  
  \begin{solution}
    \emph{Способ 1: ``Универсальный''}.
    Вспомним ещё раз методику взятия подобных интегралов, основанную на универсальной тригонометрической подстановке.
    Выражаем $\cos x$ и $\diff x$ через $\tg \frac{x}{2} \hm\equiv t$:
    \[
      \begin{aligned}
        &\cos x = \cos^2\frac{x}{2} - \sin^2\frac{x}{2} = \frac{1 - t^2}{1 + t^2}\\
        &\diff x = \diff (2\arctg t) = \frac{2 \diff t}{1 + t^2}
      \end{aligned}
    \]

    Интеграл принимает вид:
    \[
      J = \int \frac{\frac{2}{1 + t^2}}{1 + 4\frac{1 - t^2}{1 + t^2}} \diff t
      = 2\int \frac{1}{5 - 3t^2} \diff t
    \]

    И это табличный интеграл.

    \medskip

    \noindent
    \emph{Способ 2: ``Преобразование единицы и двойной половинный''}.

    Можно заметить, что есть возможность ``переписать'' знаменатель дроби как сумму/разность квадратов синуса и косинуса.
    Сделаем это (а потом посмотрим, можно ли будет с этим что-то ещё сделать дальше).

    Посмотрим на косинус как на косинус ``двойного половинного'' угла, а на единицу~---~как на сумму квадратов синуса и косинуса половинного угла:
    \[
      J = \int \frac{1}{\left(\sin^2 \frac{x}{2} + \cos^2 \frac{x}{2}\right) + 4\left(\cos^2 \frac{x}{2} - \sin^2 \frac{x}{2}\right)} \diff x
      = \int \frac{1}{5 \cos^2 \frac{x}{2} - 3 \sin^2 \frac{x}{2}} \diff x = \spadesuit
    \]

    Если теперь вынести в знаменателе за скобку $\cos^2 \frac{x}{2}$ и потом занести $\frac{1}{\cos^2 \frac{x}{2}}$ под дифференциал, то снова придём к замене $\tg \frac{x}{2} \hm\equiv t$:
    \[
      \spadesuit = 2\int \frac{1}{5 - 3 \tg ^2 \frac{x}{2}} \cdot \frac{1}{\cos^2\frac{x}{2}} \diff \left(\frac{x}{2}\right) = 2 \int \frac{1}{5 - 3t^2} \diff t
    \]
  \end{solution}
\end{document}
